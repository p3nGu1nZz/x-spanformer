\subsection{Seed Embeddings and Candidate Set}
\label{sec:seed-embeddings}

The X-Spanformer pipeline begins with a sequence \(x = [x_1, x_2, \ldots, x_T]\) of \(T\) raw Unicode codepoints. Given the induced vocabulary \(V = \{u_1, u_2, \ldots, u_{|V|}\}\) from Section~\ref{sec:vocab-induction}, we compute position-wise soft piece probabilities and transform them into contextual embeddings that serve as input to the span predictor.

\subsubsection{Soft Piece Probability Computation}
\label{sec:soft-piece-probs}

We extend the Viterbi decoding from vocabulary induction to compute soft probabilities indicating where each vocabulary piece is likely to start within the input sequence. For position \(t \in \{1, 2, \ldots, T\}\) and piece \(u_i \in V\), we define:
\begin{equation}
P_{t,i} = \Pr\bigl(\text{piece }u_i\text{ starts at position }t \mid x, V\bigr),
\label{eq:soft-piece-probability}
\end{equation}
forming the probability matrix \(P \in \mathbb{R}^{T \times |V|}\).

\paragraph{Forward-Backward Algorithm for Soft Probabilities}

To compute these probabilities efficiently, we adapt the forward-backward algorithm from HMMs to the Unigram-LM segmentation problem. Let \(\alpha_t\) denote the forward probability of reaching position \(t\), and \(\beta_t\) the backward probability from position \(t\) to the sequence end.

\textbf{Forward Pass:} Initialize \(\alpha_1 = 1\) and compute recursively:
\begin{equation}
\alpha_{t+1} = \sum_{u_i \in V : \text{match}(x, t, u_i)} \alpha_t \cdot p(u_i),
\label{eq:forward-pass}
\end{equation}
where \(\text{match}(x, t, u_i)\) indicates whether piece \(u_i\) matches the substring starting at position \(t\).

\textbf{Backward Pass:} Initialize \(\beta_{T+1} = 1\) and compute recursively:
\begin{equation}
\beta_t = \sum_{u_i \in V : \text{match}(x, t, u_i)} p(u_i) \cdot \beta_{t + |u_i|},
\label{eq:backward-pass}
\end{equation}
where \(|u_i|\) denotes the length of piece \(u_i\) in codepoints.

\textbf{Soft Probability Calculation:} The soft probability for piece \(u_i\) starting at position \(t\) is:
\begin{equation}
P_{t,i} = \frac{\alpha_t \cdot p(u_i) \cdot \beta_{t + |u_i|}}{\alpha_{T+1}},
\quad \text{if } \text{match}(x, t, u_i), \text{ else } 0.
\label{eq:soft-probability}
\end{equation}

This formulation ensures that \(\sum_{i=1}^{|V|} P_{t,i} \leq 1\) for all positions \(t\), with equality when every position is covered by at least one piece.

\subsubsection{Embedding Matrix Initialization and Seed Computation}
\label{sec:seed-embeddings-computation}

The embedding matrix \(W_{\mathrm{emb}} \in \mathbb{R}^{|V| \times d}\) projects vocabulary pieces into a dense \(d\)-dimensional representation space. Each row \(W_{\mathrm{emb}}[i, :]\) corresponds to the embedding vector for piece \(u_i \in V\).

\paragraph{Initialization Strategy}

We initialize \(W_{\mathrm{emb}}\) using vocabulary-aware strategies that leverage the statistical properties from Section~\ref{sec:vocab-induction}:

\begin{align}
W_{\mathrm{emb}}[i, :] &\sim \mathcal{N}\left(0, \frac{\sigma^2}{\sqrt{p(u_i)}}\right) \\
&\quad \text{(frequency-scaled Gaussian)} \label{eq:freq-scaled-init} \\
\text{where } \sigma^2 &= \frac{2}{d + |V|} \quad \text{(Xavier-style scaling)} \label{eq:xavier-scaling}
\end{align}

This initialization gives more stable gradients to frequent pieces (higher \(p(u_i)\)) while maintaining overall variance control. 

Single codepoints, which have guaranteed non-zero probability, receive standard Xavier initialization.

\paragraph{Seed Embedding Computation}

The initial seed embeddings are computed via matrix multiplication:
\begin{equation}
H^0 = P \cdot W_{\mathrm{emb}} \in \mathbb{R}^{T \times d},
\label{eq:seed-embeddings}
\end{equation}
where each row \(H^0[t, :]\) represents the weighted combination of piece embeddings that are likely to start at position \(t\). 

This soft aggregation preserves uncertainty from the probabilistic segmentation while providing a dense representation for downstream processing.

\subsubsection{Contextual Encoder: Multi-Scale Dilated Convolutions}
\label{sec:contextual-encoder}

To enrich the seed embeddings with local contextual information, we apply a multi-scale dilated convolutional encoder that captures compositional patterns across different length scales. The encoder transforms \(H^0 \in \mathbb{R}^{T \times d}\) into contextualized representations \(H \in \mathbb{R}^{T \times d}\):
\begin{equation}
H = \mathrm{ConvEncoder}(H^0),\quad \text{computational cost } O(T d^2).
\label{eq:conv-encoder}
\end{equation}

\paragraph{Multi-Scale Architecture}

The encoder employs three parallel convolutional pathways with different kernel sizes \(K = \{3, 5, 7\}\) and dilation rates \(D = \{1, 2, 4\}\), creating a total of \(|K| \times |D| = 9\) distinct receptive field patterns. For kernel size \(k \in K\) and dilation \(d \in D\), the effective receptive field spans:
\begin{equation}
\text{RF}_{k,d} = 1 + (k-1) \cdot d \quad \text{positions},
\label{eq:receptive-field}
\end{equation}
yielding receptive fields from 3~positions (k=3, d=1) to 25~positions (k=7, d=4).

\paragraph{Dilated Convolution Formulation}

For a single pathway with kernel size \(k\) and dilation \(d\), the convolution at position \(t\) with input \(H^{(\ell)} \in \mathbb{R}^{T \times d_{\ell}}\) and weight tensor \(W^{(k,d)} \in \mathbb{R}^{k \times d_{\ell} \times d_{\ell+1}}\) is:
\begin{equation}
H^{(\ell+1)}[t, :] = \sum_{j=0}^{k-1} W^{(k,d)}[j, :, :] \cdot H^{(\ell)}[t - j \cdot d, :] + b^{(k,d)},
\label{eq:dilated-conv}
\end{equation}
where boundary conditions apply zero-padding for out-of-bounds indices.

\paragraph{Pathway Fusion and Residual Connections}

The outputs from all \(|K| \times |D|\) pathways are concatenated and projected back to dimension \(d\):
\begin{align}
H^{(\text{concat})} &= \text{Concat}\left(\left\{H^{(k,d)}\right\}_{k \in K, d \in D}\right) \nonumber \\
&\quad \in \mathbb{R}^{T \times (|K| \cdot |D| \cdot d)} \\
H^{(\text{proj})} &= H^{(\text{concat})} \cdot W_{\text{proj}} + b_{\text{proj}} \in \mathbb{R}^{T \times d} \\
H &= \text{LayerNorm}(H^0 + H^{(\text{proj})}) \quad \text{(residual connection)}
\label{eq:pathway-fusion}
\end{align}

This residual design ensures that the encoder preserves the original seed embedding information while adding contextual refinements.

\begin{proposition}[Computational Complexity]
	The multi-scale dilated convolutional encoder has computational complexity \(O(T d^2 \cdot |K| \cdot |D|)\) and memory complexity \(O(T d)\) for a sequence of length \(T\).
\end{proposition}

\begin{proof}
	For each of the \(|K| \cdot |D|\) pathways, the dilated convolution operation requires \(O(T \cdot k \cdot d^2)\) operations, where \(k\) is the kernel size. 
	
	Summing over all pathways:
	\begin{align}
	\sum_{k \in K, d \in D} O(T \cdot k \cdot d^2) &= O\left(T d^2 \sum_{k \in K} k \cdot |D|\right) \nonumber \\
	&= O(T d^2 \cdot |K| \cdot |D|)
	\label{eq:conv-complexity}
	\end{align}
	since \(\sum_{k \in K} k = O(|K|)\) for fixed kernel sizes.
	
	The concatenation and projection steps add \(O(T \cdot |K| \cdot |D| \cdot d^2)\) operations, which is dominated by the convolution cost.
	
	Memory complexity is \(O(T d)\) since intermediate activations are computed sequentially and the final output has the same dimensions as the input.
\end{proof}

\paragraph{Architectural Rationale}

This convolutional design choice reflects several architectural priorities:

\begin{itemize}
	\item \textbf{Linear Scaling:} Unlike quadratic self-attention, convolution scales linearly with sequence length, enabling processing of long documents and code files.
	\item \textbf{Local Compositionality:} Multi-scale kernels capture hierarchical patterns relevant to span boundaries, from character clusters to word-level units.
	\item \textbf{Positional Inductive Bias:} Convolutions preserve spatial relationships critical for segmentation tasks, maintaining awareness of relative positions within spans.
	\item \textbf{Interpretability:} Individual pathways can be analyzed to understand which length scales contribute most to span predictions.
\end{itemize}

\subsubsection{Span Candidate Enumeration and Filtering}
\label{sec:span-candidates}

Given the contextualized sequence \(H \in \mathbb{R}^{T \times d}\), we enumerate contiguous span candidates that serve as input to the boundary prediction module. The candidate generation process balances completeness with computational efficiency through width-based filtering and structural constraints.

\paragraph{Basic Candidate Generation}

We define the initial candidate set as all contiguous subsequences within the maximum span width:
\begin{equation}
C_{\text{raw}} = \{(i,j) \mid 1 \leq i < j \leq T, \; j - i + 1 \leq w_{\max}\},
\label{eq:raw-candidates}
\end{equation}
where \(w_{\max}\) is the maximum allowed span width in codepoints. This constraint reduces the candidate space from quadratic \(O(T^2)\) to linear \(O(T \cdot w_{\max})\).

\paragraph{Vocabulary-Informed Filtering}

To focus computation on linguistically plausible spans, we apply additional filtering based on the induced vocabulary structure. A span \((i,j)\) is retained if it satisfies at least one of the following criteria:

\begin{enumerate}
	\item \textbf{Vocabulary Alignment:} The span corresponds to a high-probability piece:
	\begin{equation}
	\exists \, u_k \in V : \text{span}(x, i, j) = u_k \wedge p(u_k) \geq \tau_{\text{vocab}}
	\label{eq:vocab-alignment}
	\end{equation}
	
	\item \textbf{Compositional Potential:} The span can be segmented into multiple vocabulary pieces:
	\begin{equation}
	\exists \, \text{seg} \in \text{Segments}(x, i, j) : \prod_{u \in \text{seg}} p(u) \geq \tau_{\text{comp}}
	\label{eq:compositional-potential}
	\end{equation}
	
	\item \textbf{Boundary Coherence:} The span respects whitespace separation constraints from Section~\ref{sec:whitespace-tokenization}:
	\begin{equation}
	\text{WhitespaceCoherent}(\text{span}(x, i, j)) = \text{True}
	\label{eq:boundary-coherence}
	\end{equation}
\end{enumerate}

The filtered candidate set becomes:
\begin{equation}
C = \{(i,j) \in C_{\text{raw}} \mid \text{satisfies at least one criterion above}\}.
\label{eq:filtered-candidates}
\end{equation}

\paragraph{Efficient Candidate Storage}

For implementation efficiency, candidates are stored as a sparse tensor \(\mathcal{C} \in \{0,1\}^{T \times T}\) where \(\mathcal{C}[i,j] = 1\) if \((i,j) \in C\) and 0 otherwise. This representation enables:

\begin{itemize}
	\item \textbf{Batch Processing:} Multiple sequences can share the same indexing structure
	\item \textbf{Memory Efficiency:} Only valid spans consume storage via sparse tensor operations
	\item \textbf{GPU Acceleration:} Parallel span evaluation using tensor broadcasting
\end{itemize}

\begin{algorithm}[H]
	\caption{Span Candidate Enumeration}
	\label{alg:span-candidates}
	\begin{algorithmic}[1]
		\STATE \textbf{Input:} Sequence length \(T\), vocabulary \(V\), probabilities \(\{p(u)\}\), 
		\STATE \qquad thresholds \(\tau_{\text{vocab}}, \tau_{\text{comp}}\)
		\STATE Initialize sparse tensor \(\mathcal{C} \in \{0,1\}^{T \times T}\) with zeros
		\FOR{\(i = 1\) to \(T-1\)}
		\FOR{\(j = i+1\) to \(\min(i + w_{\max}, T)\)}
		\STATE \(s \leftarrow \text{span}(x, i, j)\) \COMMENT{Extract span text}
		\IF{VocabularyAlignment\((s, V, \tau_{\text{vocab}})\)}
		\STATE \(\mathcal{C}[i,j] \leftarrow 1\)
		\ELSIF{CompositionalPotential\((s, V, \tau_{\text{comp}})\)}
		\STATE \(\mathcal{C}[i,j] \leftarrow 1\)
		\ELSIF{WhitespaceCoherent\((s)\)}
		\STATE \(\mathcal{C}[i,j] \leftarrow 1\)
		\ENDIF
		\ENDFOR
		\ENDFOR
		\STATE \textbf{Return:} Candidate tensor \(\mathcal{C}\)
	\end{algorithmic}
\end{algorithm}

\begin{proposition}[Candidate Set Completeness]
	For any vocabulary \(V\) containing all single codepoints, the filtered candidate set \(C\) contains all spans that can be perfectly segmented under \(V\).
\end{proposition}

\begin{proof}
	Let \((i,j)\) be any span such that \(\text{span}(x, i, j)\) can be perfectly segmented using pieces from \(V\). We consider two cases:
	
	\textbf{Case 1:} The entire span corresponds to a single piece \(u \in V\). Then by vocabulary alignment criterion (1), \((i,j) \in C\) whenever \(p(u) \geq \tau_{\text{vocab}}\).
	
	\textbf{Case 2:} The span requires multiple pieces. Let \(\text{seg} = [u_1, u_2, \ldots, u_k]\) be the optimal segmentation. Since each \(u_i \in V\) has non-zero probability from the EM training in Section~\ref{sec:vocab-induction}, we have \(\prod_{i=1}^k p(u_i) > 0\). For sufficiently small \(\tau_{\text{comp}}\), this satisfies criterion (2), so \((i,j) \in C\).
	
	\textbf{Boundary Case:} If the span violates whitespace coherence but satisfies segmentation criteria, it may still be included via criteria (1) or (2), ensuring no valid linguistic spans are excluded.
\end{proof}

This candidate enumeration provides the foundation for downstream span boundary prediction while maintaining computational tractability through principled filtering based on the statistical properties learned during vocabulary induction.

\subsubsection{Pipeline Integration and Algorithm Synthesis}
\label{sec:pipeline-integration}

The complete seed embedding and candidate generation pipeline integrates the vocabulary induction from Section~\ref{sec:vocab-induction} with the contextual encoding and candidate filtering described above. This section presents the unified algorithm and establishes computational guarantees.

\paragraph{End-to-End Pipeline}

Given raw codepoint sequence \(x = [x_1, \ldots, x_T]\) and induced vocabulary \(V\) with probabilities \(\{p(u)\}_{u \in V}\), the complete pipeline produces:

\begin{enumerate}
	\item \textbf{Soft probabilities} \(P \in \mathbb{R}^{T \times |V|}\) via forward-backward algorithm
	\item \textbf{Seed embeddings} \(H^0 = P \cdot W_{\mathrm{emb}} \in \mathbb{R}^{T \times d}\)  
	\item \textbf{Contextual embeddings} \(H = \text{ConvEncoder}(H^0) \in \mathbb{R}^{T \times d}\)
	\item \textbf{Span candidates} \(C = \{(i,j)\}\) with vocabulary-informed filtering
\end{enumerate}

\begin{algorithm}[H]
	\caption{Unified Seed Embedding and Candidate Generation}
	\label{alg:seed-embedding-pipeline}
	\begin{algorithmic}[1]
		\STATE \textbf{Input:} Codepoint sequence \(x \in \mathbb{Z}^T\), vocabulary \(V\), 
		\STATE \qquad piece probabilities \(\{p(u)\}\)
		\STATE \textbf{Parameters:} Embedding matrix \(W_{\mathrm{emb}} \in \mathbb{R}^{|V| \times d}\), 
		\STATE \qquad conv weights, span width \(w_{\max}\)
		\STATE
		\STATE \textbf{// Step 1: Soft Probability Computation}
		\STATE Initialize forward probabilities: \(\alpha_1 \leftarrow 1\), 
		\STATE \qquad \(\alpha_t \leftarrow 0\) for \(t > 1\)
		\FOR{\(t = 1\) to \(T\)}
		\FOR{each piece \(u_i \in V\) where \(\text{match}(x, t, u_i)\)}
		\STATE \(\alpha_{t + |u_i|} \leftarrow \alpha_{t + |u_i|} + \alpha_t \cdot p(u_i)\)
		\ENDFOR
		\ENDFOR
		\STATE
		\STATE Initialize backward probabilities: \(\beta_{T+1} \leftarrow 1\), 
		\STATE \qquad \(\beta_t \leftarrow 0\) for \(t \leq T\)
		\FOR{\(t = T\) down to \(1\)}
		\FOR{each piece \(u_i \in V\) where \(\text{match}(x, t, u_i)\)}
		\STATE \(\beta_t \leftarrow \beta_t + p(u_i) \cdot \beta_{t + |u_i|}\)
		\ENDFOR
		\ENDFOR
		\STATE
		\STATE \textbf{// Step 2: Construct Probability Matrix}
		\FOR{\(t = 1\) to \(T\)}
		\FOR{\(i = 1\) to \(|V|\)}
		\IF{\(\text{match}(x, t, u_i)\)}
		\STATE \(P[t, i] \leftarrow \frac{\alpha_t \cdot p(u_i) \cdot \beta_{t + |u_i|}}{\alpha_{T+1}}\)
		\ELSE
		\STATE \(P[t, i] \leftarrow 0\)
		\ENDIF
		\ENDFOR
		\ENDFOR
		\STATE
		\STATE \textbf{// Step 3: Embedding and Contextualization}
		\STATE \(H^0 \leftarrow P \cdot W_{\mathrm{emb}}\) \COMMENT{Seed embeddings}
		\STATE \(H \leftarrow \text{ConvEncoder}(H^0)\) \COMMENT{Multi-scale contextualization}
		\STATE
		\STATE \textbf{// Step 4: Candidate Generation}
		\STATE \(C \leftarrow \text{EnumerateCandidates}(x, V, w_{\max})\) 
		\STATE \qquad \COMMENT{Algorithm~\ref{alg:span-candidates}}
		\STATE
		\STATE \textbf{Return:} Embeddings \(H\), candidates \(C\), probabilities \(P\)
	\end{algorithmic}
\end{algorithm}

\begin{proposition}[Pipeline Computational Complexity]
	The unified pipeline has time complexity \(O(T \cdot |V| \cdot L_{\max} + T \cdot d^2 + T \cdot w_{\max}^2)\) and space complexity \(O(T \cdot |V| + T \cdot d)\), where \(L_{\max}\) is the maximum piece length.
\end{proposition}

\begin{proof}
	We analyze each step separately:
	
	\textbf{Step 1 (Forward-Backward):} For each position \(t\), we check matching against all pieces \(u_i \in V\), each taking \(O(|u_i|) = O(L_{\max})\) time. 
	
	Total: \(O(T \cdot |V| \cdot L_{\max})\).
	
	\textbf{Step 2 (Probability Matrix):} Direct computation for \(T \times |V|\) entries: \(O(T \cdot |V|)\).
	
	\textbf{Step 3 (Embeddings):} Matrix multiplication \(P \cdot W_{\mathrm{emb}}\) costs \(O(T \cdot |V| \cdot d)\), and convolution costs \(O(T \cdot d^2)\).
	
	\textbf{Step 4 (Candidates):} Enumerating \(O(T \cdot w_{\max})\) candidates, each requiring \(O(w_{\max})\) validation: \(O(T \cdot w_{\max}^2)\).
	
	The dominant terms are \(O(T \cdot |V| \cdot L_{\max})\) for probability computation and \(O(T \cdot d^2)\) for contextualization.
	
	Space complexity is dominated by the probability matrix \(P \in \mathbb{R}^{T \times |V|}\) and contextual embeddings \(H \in \mathbb{R}^{T \times d}\).
\end{proof}

\paragraph{Implementation Considerations}

The pipeline design enables several practical optimizations:

\begin{itemize}
	\item \textbf{Batch Processing:} Multiple sequences can share embedding matrices and convolution operations
	\item \textbf{Memory Streaming:} Probability matrices can be computed in chunks for very long sequences
	\item \textbf{GPU Acceleration:} All matrix operations are amenable to parallel execution
	\item \textbf{Sparse Computation:} Many \(P[t,i]\) entries are zero due to vocabulary mismatch, enabling sparse tensor optimizations
\end{itemize}

This integrated pipeline transforms raw codepoint sequences into structured representations suitable for span-aware downstream processing, while maintaining linear scaling properties and preserving the statistical foundations established during vocabulary induction.