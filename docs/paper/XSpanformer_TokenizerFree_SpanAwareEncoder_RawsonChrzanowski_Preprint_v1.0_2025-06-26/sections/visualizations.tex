\section{Visualization Framework and Interpretability Interfaces}
\label{sec:visualizations}

Interpretability is central to the X-Spanformer framework, not only for debugging but for validating the emergence of structured behavior from differentiable routing. We introduce a modular visualization suite designed to capture dynamic routing patterns, evaluate alignment with linguistic structure, and probe causal effects of controller activations.

These interfaces build on the interpretability literature in structured attention \cite{vig2019analyzing, hoover2020exbert}, entropy-based pruning \cite{grandvalet2006entropy, pereyra2017regularizing}, and latent syntactic probing \cite{linzen2016assessing, kim2019unsupervised}. Together, they scaffold an interactive environment for qualitative and quantitative analysis of controller behavior throughout training.

\subsection{Span Trajectory Viewer (\texttt{trajectory\_overlay})}
\label{sec:vis-traj}

The span trajectory viewer\footnote{Implemented as an interactive Plotly dashboard with real-time epoch selection and layer filtering capabilities.} highlights how span selection stabilizes or evolves over training epochs. For a given input prompt, we track the top-$K$ spans selected at each layer \( \ell \) and epoch \( t \), plotting their token offsets and confidence weights \( \alpha_k^{(t)} \). This provides a temporal window into routing stability and sparsification behavior.

\subsubsection*{Method}

\begin{enumerate}[leftmargin=1.5em]
  \item Cache span logits \( w_k^{(t)} \) and compute attention weights \( \alpha_k^{(t)} = \mathrm{softmax}(w_k^{(t)}) \)
  \item Compute per-epoch entropy:
  \begin{equation}
    H(P_t) = - \sum_k \alpha_k^{(t)} \log \alpha_k^{(t)}
    \label{eq:entropy_viz}
  \end{equation}
  \item Compute inter-epoch routing divergence using Jensen-Shannon divergence for numerical stability:
  \begin{equation}
    D_{\mathrm{JS}}(P_t \| P_{t+1}) = \frac{1}{2} \left[ \mathrm{KL}(P_t \| M) + \mathrm{KL}(P_{t+1} \| M) \right]
    \label{eq:jsdiv}
  \end{equation}
  where \(M = \frac{1}{2}(P_t + P_{t+1})\) is the mixture distribution.
  
  \textbf{Mathematical Properties of Jensen-Shannon Divergence:}
  \begin{enumerate}[leftmargin=1.5em]
  \item \emph{Symmetry:} \(D_{\mathrm{JS}}(P \| Q) = D_{\mathrm{JS}}(Q \| P)\)
  \item \emph{Boundedness:} \(0 \le D_{\mathrm{JS}}(P \| Q) \le \log 2\) for probability distributions
  \item \emph{Metric Property:} \(\sqrt{D_{\mathrm{JS}}(P \| Q)}\) satisfies the triangle inequality
  \end{enumerate}
  \item Render span overlays layerwise with bar intensity mapped to \( \alpha_k^{(t)} \)
\end{enumerate}

This supports empirical validation of our entropy convergence claim from Proposition~\ref{prop:annealing}, echoing trends found in routing-based sparse architectures \cite{tay2020sparse, shazeer2017outrageously}.

\subsection{Span Alignment Grid (\texttt{span\_grid\_align})}
\label{sec:vis-align}

To assess whether X-Spanformer’s induced spans align with latent syntax or semantics, we compute overlap heatmaps comparing model-predicted spans to gold annotations from syntactic and semantic corpora.

\subsubsection*{Method}

\begin{enumerate}[leftmargin=1.5em]
  \item Extract top-$K$ spans \( \{(i_k, j_k)\} \) from controller at layer \( \ell \)
  \item Align with:
  \begin{itemize}
    \item Constituents: Berkeley parser\footnote{Berkeley Neural Parser trained on Penn Treebank WSJ sections 02-21, achieving 95.8\% F1 on constituency parsing.} \cite{kitaev2018constituency}
    \item Named entities: SpaCy\footnote{SpaCy v3.4 English model (en\_core\_web\_sm) with pre-trained NER components.} \cite{honnibal2017spacy}
    \item Discourse units: OntoNotes\footnote{OntoNotes 5.0 coreference and semantic role labeling annotations.} \cite{weischedel2013ontonotes}
  \end{itemize}
  \item Compute Jaccard index \( J(s_k, r_j) = \frac{|s_k \cap r_j|}{|s_k \cup r_j|} \)
  \item Generate token-layer grid showing alignment scores
\end{enumerate}

This grid supports span-level interpretability claims from Section~\ref{sec:qualitative-spans}, complementing prior syntactic induction probes \cite{kim2019unsupervised}.

\subsection{Controller Influence Map (\texttt{influence\_heatmap})}
\label{sec:vis-influence}

This module evaluates the downstream sensitivity of network outputs to variation in the controller signal \( \tilde{s} \). We use this to assess disentanglement and directional salience of span-based routing.

\subsubsection*{Method}

\begin{enumerate}[leftmargin=1.5em]
  \item Inject perturbed vectors \( \tilde{s}_\text{baseline} \), \( \tilde{s}_\text{perturbed} \) into layer \( \ell \)
  \item Measure output response via the \(L_2\) norm of output differences:
  \begin{equation}
    \delta_{\text{out}} = \left\| \mathcal{F}(x, \tilde{s}_1) - \mathcal{F}(x, \tilde{s}_2) \right\|_2
    \label{eq:controller_effect}
  \end{equation}
  where \(\mathcal{F}(x, \tilde{s})\) represents the transformer output given input \(x\) and controller vector \(\tilde{s}\).
  
  \textbf{Interpretation of Controller Influence:}
  \begin{itemize}[leftmargin=1.5em]
  \item \(\delta_{\text{out}} = 0\): No sensitivity to controller perturbation
  \item \(\delta_{\text{out}} \gg 0\): High sensitivity, indicating strong controller influence
  \item Normalized version: \(\delta_{\text{norm}} = \frac{\delta_{\text{out}}}{\|\mathcal{F}(x, \tilde{s}_1)\|_2}\) provides scale-invariant interpretation
  \end{itemize}
  \item Visualize effect across token positions and logit spaces
\end{enumerate}

This is inspired by mediation analyses in causal probing \cite{vig2020investigating, belinkov2019analyzing} and offers interpretability of routing pathways without direct supervision.

\subsection{Entropy Field Morphometry (\texttt{entropy\_map})}
\label{sec:vis-entropy}

To inspect global routing structure, we visualize span entropy across token windows and layers. This “morphometric map” reveals compositional boundaries, entropy basins, and emergence of high-confidence foci.

\subsubsection*{Method}

\begin{enumerate}[leftmargin=1.5em]
  \item For every candidate span \( (i,j) \) at each layer, compute:
  \begin{equation}
    H_{i:j}^{(\ell)} = -\sum_k \alpha_k^{(\ell)} \log \alpha_k^{(\ell)}, \quad \text{where } \alpha_k^{(\ell)} \text{ selects spans overlapping } (i, j)
    \label{eq:entropy_local}
  \end{equation}
  \item Aggregate per-position entropy into a 2D token-layer grid
  \item Render high-confidence routes as brightness troughs
\end{enumerate}

This is modeled after flow-field visualizations in neural saliency \cite{olah2018building}, adapted here for differentiable sparse span selectors \cite{pereyra2017regularizing, liu2024structured}.
