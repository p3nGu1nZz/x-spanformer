\subsection{Span Induction with Entropic Regularization}
\label{sec:span-induction}

Starting from contextualized embeddings \(H\in\mathbb{R}^{T\times d}\), we define the set of contiguous spans of maximum width \(w_{\max}\):
\begin{equation}
	S = \bigl\{(i,j)\;\big|\;1\le i<j\le \min(i+w_{\max},T)\bigr\}.
	\label{eq:contiguous_span_set}
\end{equation}
Each span \((i,j)\in S\) is pooled to a fixed-length vector
\[
v_{ij} = \mathrm{Pool}\bigl(H[i:j]\bigr)\in\mathbb{R}^d,
\]
and scored by a small induction network \(f_{\mathrm{ind}}:\mathbb{R}^d\to\mathbb{R}\):
\[
w_{ij} = f_{\mathrm{ind}}(v_{ij}).
\]
We normalize these logits to obtain a distribution over spans:
\begin{equation}
	P_{ij}
	= \frac{\exp(w_{ij})}
	{\sum_{(a,b)\in S}\exp(w_{ab})}.
	\label{eq:span_softmax}
\end{equation}

To prevent early collapse into a few high–confidence spans, we add a temperature–weighted entropy penalty:
\begin{equation}
	\mathcal{L}_{\mathrm{ent}}
	= -\lambda_{\mathrm{ent}}(t)\;H(P),
	\quad
	H(P) = -\sum_{(i,j)\in S}P_{ij}\log P_{ij},
	\label{eq:entropy_term}
\end{equation}
with an annealing schedule
\begin{equation}
	\lambda_{\mathrm{ent}}(t)
	= \lambda_0\,e^{-\gamma t},
	\label{eq:entropy_decay}
\end{equation}
where \(t\) indexes training epochs, \(\lambda_0>0\) is the initial weight, and \(\gamma>0\) the decay rate.  This follows entropy regularization principles \cite{grandvalet2005semi,pereyra2017regularizing} and curriculum learning schedules \cite{bengio2009curriculum,kreutzer2021distilling}.

\begin{proposition}[Maximum Entropy of Uniform Span Distribution]
	\label{prop:span_entropy_bound}
	Let \(|S|=N\).  Then the entropy \(H(P)\) in Equation~\eqref{eq:entropy_term} is maximized when
	\begin{equation}
		P_{ij} = \frac{1}{N}
		\quad\forall\,(i,j)\in S,
		\label{eq:uniform_P}
	\end{equation}
	yielding
	\begin{equation}
		H_{\max}(P) = \log N.
		\label{eq:max_entropy}
	\end{equation}
\end{proposition}
\begin{proof}
	We maximize the entropy function
	\[
	H(P) = -\sum_{(i,j)\in S} P_{ij} \log P_{ij}
	\]
	subject to the normalization constraint \(\sum_{(i,j)\in S} P_{ij} = 1\) and non-negativity \(P_{ij} \geq 0\).
	
	\textbf{Step 1:} Construct the Lagrangian functional.
	\[
	\mathcal{L}(P, \lambda) = -\sum_{(i,j)\in S} P_{ij} \log P_{ij} + \lambda\left(\sum_{(i,j)\in S} P_{ij} - 1\right)
	\]
	
	\textbf{Step 2:} Compute the first-order optimality condition for \(P_{kl}\) where \((k,l) \in S\).
	\[
	\frac{\partial \mathcal{L}}{\partial P_{kl}} = -\log P_{kl} - 1 + \lambda = 0
	\]
	
	\textbf{Step 3:} Solve for \(P_{kl}\).
	\[
	\log P_{kl} = \lambda - 1 \quad \Rightarrow \quad P_{kl} = e^{\lambda - 1}
	\]
	
	\textbf{Step 4:} Since this relation holds for every \((k,l) \in S\), all probabilities are equal: \(P_{ij} = c\) for some constant \(c = e^{\lambda - 1}\).
	
	\textbf{Step 5:} Apply the normalization constraint.
	\[
	\sum_{(i,j)\in S} P_{ij} = \sum_{(i,j)\in S} c = |S| \cdot c = N \cdot c = 1
	\]
	Therefore, \(c = \frac{1}{N}\), which gives \(P_{ij} = \frac{1}{N}\) for all \((i,j) \in S\).
	
	\textbf{Step 6:} Substitute into the entropy expression.
	\[
	H_{\max}(P) = -\sum_{(i,j)\in S} \frac{1}{N} \log\left(\frac{1}{N}\right) = -\frac{N}{N} \log\left(\frac{1}{N}\right) = -\log\left(\frac{1}{N}\right) = \log N
	\]
\end{proof}

\noindent\textbf{Remark.} Early in training, a high \(\lambda_{\mathrm{ent}}(t)\) encourages broad span exploration.  As \(\lambda_{\mathrm{ent}}(t)\) decays, the model concentrates probability mass on a sparse set of high‐salience spans, facilitating convergence to meaningful structural units.
