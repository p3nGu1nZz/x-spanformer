\subsection{End-to-End Fine-Tuning}
\label{sec:end-to-end-finetuning}

After the span scorer \(f_\theta\) and aggregator \(g_\phi\) have learned stable inductive patterns, we integrate the fused controller \(\tilde{s}\) into the transformer backbone \(\psi\) and optimize all components jointly.

\subsubsection{Joint Routing and Regularization}
\label{sec:joint-routing-regularization}

The total objective over epochs \(t=T_1+1,\dots,T_2\) is

\begin{equation}
	\mathcal{L}_{\mathrm{total}}(t)
	= \mathcal{L}_{\mathrm{task}}
	+ \beta_{1}\,\mathcal{L}_{\mathrm{span}}
	+ \beta_{2}\,\lambda_{\mathrm{ent}}(t)\,H(P),
	\label{eq:curriculum_total}
\end{equation}

where
\(\mathcal{L}_{\mathrm{task}}\) is the downstream loss (e.g., negative log‐likelihood, classification cross‐entropy, or contrastive objective),  
\(\mathcal{L}_{\mathrm{span}} = \mathrm{KL}\bigl(P_{\mathrm{gold}}\Vert P_{\theta}\bigr)\) aligns the induced span distribution \(P_\theta\) with any available span supervision, and  
\(H(P)\) is the Shannon entropy of \(P_\theta\).  
The entropy coefficient is annealed only after the Phase I transition epoch \(T_1\):
\begin{equation}
\lambda_{\mathrm{ent}}(t)
= \lambda_{0}\,\exp\bigl(-\gamma\,(t - T_1)\bigr)\,\mathbf{1}_{t > T_1},
\label{eq:entropy_schedule}
\end{equation}
ensuring continued but diminishing exploration during fine‐tuning.

\textbf{Mathematical Properties of the Annealing Schedule:}
\begin{enumerate}[leftmargin=1.5em]
\item \emph{Continuity at Transition:} \(\lim_{t \to T_1^+} \lambda_{\mathrm{ent}}(t) = \lambda_0\)
\item \emph{Monotonic Decay:} For \(t > T_1\), \(\frac{d\lambda_{\mathrm{ent}}}{dt} = -\gamma \lambda_0 \exp(-\gamma(t-T_1)) < 0\)
\item \emph{Asymptotic Behavior:} \(\lim_{t \to \infty} \lambda_{\mathrm{ent}}(t) = 0\)
\end{enumerate}
This exponential decay balances initial exploration (\(\lambda_0\)) with eventual exploitation (convergence to 0).

\subsubsection{Training Algorithm}
\label{sec:training-algorithm}

\medskip
\noindent\textbf{Algorithm: Joint Optimization}

\begin{algorithm}[H]
	\caption{End‐to‐End Fine‐Tuning}
	\label{alg:e2e_finetuning}
	\begin{algorithmic}[1]
		\REQUIRE Pretrained scorer \(f_\theta\), aggregator \(g_\phi\), transformer \(\psi\)
		\FOR{epoch \(t=T_1+1\) to \(T_2\)}
		\FOR{each batch \((x,y)\)}
		\STATE Compute contextual embeddings \(H = \psi_{\mathrm{enc}}(x)\)
		\STATE Enumerate spans \(S\) and pool \(v_{ij} = \mathrm{Pool}(H[i:j])\)
		\STATE Induce span logits \(w_{ij}=f_\theta(v_{ij})\), normalize 
		\(P_{ij}=\softmax(w)\)
		\STATE Select top-\(K\) spans \(\{s_k\}\) and compute \(\tilde{s}=\sum_k a_k s_k\)
		\STATE Inject \(\tilde{s}\) into \(\psi\) via prefix/bias/gate
		\STATE Compute \(\mathcal{L}_{\mathrm{total}}(t)\) by Eq.~\eqref{eq:curriculum_total}
		\STATE Backpropagate and update \(\theta,\phi,\psi\)
		\ENDFOR
		\ENDFOR
	\end{algorithmic}
\end{algorithm}

\noindent\textbf{Summary.} This joint training approach combines span induction (\(f_\theta\)), aggregation (\(g_\phi\)), and the transformer parameters \(\psi\), using the controller vector \(\tilde{s}\) as a structural bottleneck.  This combined optimization:
\begin{itemize}
	\item Preserves high‐entropy exploration early in fine‐tuning,  
	\item Gradually shifts focus to task‐relevant spans via entropy annealing,  
	\item Learns to route structural information into the encoder through differentiable injection modes.
\end{itemize}
Empirically, this two‐stage curriculum yields stable convergence under sparse span supervision and enhances interpretability of transformer behavior \cite{belinkov2022probing}.

\subsubsection{Optimization Details}
\label{sec:optimization-details}

\medskip
\noindent We train all parameters with AdamW \cite{loshchilov2019decoupled}, using:
\begin{itemize}[leftmargin=1.5em]
	\item Cosine learning‐rate schedule with 10\% warmup,
	\item Gradient clipping at \(\|\nabla\|_2\le1.0\),
	\item Dropout rate 0.1,
	\item Batch size 64 (token‐aligned).
\end{itemize}
Full hyperparameter ranges and ablation settings are provided in Appendix~\ref{sec:hyperparams} and \ref{sec:ablation-settings}.
