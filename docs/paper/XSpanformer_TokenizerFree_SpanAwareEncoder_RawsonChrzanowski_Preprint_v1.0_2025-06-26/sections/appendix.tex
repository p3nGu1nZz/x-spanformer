\appendix
\section*{Appendix}
\addcontentsline{toc}{section}{Appendix}

\subsection{Training Hyperparameters}
\label{sec:hyperparams}

\begin{table}[H]
	\centering
	\begin{tabular}{@{}lcc@{}}
		\toprule
		Parameter & Value & Description \\
		\midrule
		Optimizer & AdamW & with decoupled weight decay \\
		Learning rate schedule & Cosine decay & with 10\% warmup \\
		Initial LR & 1e-4 & base LR used for all modules \\
		Dropout & 0.1 & applied to all nonlinearity layers \\
		Max grad norm & 1.0 & gradient clipping threshold \\
		Epochs & 50 & full fine-tuning duration \\
		Batch size & 64 & across all stages \\
		Span width $w_{\max}$ & 10 & max width considered per token \\
		Entropy $\lambda_0$ & 1.0 & initial entropy coefficient \\
		Decay $\gamma$ & 0.1 & exponential decay rate \\
		Span pooling strategy & Gated self-attention & with key-query masking and layer norm \\
		\bottomrule
	\end{tabular}
	\caption{Hyper-parameters used in all experiments. Span embeddings are pooled using \texttt{Pool}$(x_{i:j})$, which may implement mean, max, or gated self-attention over the selected token embeddings.}
\end{table}

\subsection{Additional Experimental Details}
\label{sec:extra-exp}

\begin{itemize}[leftmargin=1.5em]
	\item All models trained on a single NVIDIA A100 GPU.
	\item Training time per epoch ranged from 1.1 to 2.3 minutes, depending on sequence length.
	\item Random seeds and full configuration files will be released alongside the code for exact reproducibility.
\end{itemize}

\subsection{Extended Ablation Settings}
\label{sec:ablation-settings}

\begin{itemize}[leftmargin=1.5em]
	\item \textbf{Fusion head variants}: Compared \texttt{MLP(w)} vs.\ \texttt{LayerNorm(MLP)} for $\alpha_k$ scoring; gated units improved stability in low-entropy regimes.
	\item \textbf{Routing depth}: Explored controller depth $d_c \in \{1,2,3\}$; performance plateaued beyond $d_c=2$.
	\item \textbf{Gradient gating}: Evaluated freezing the span scorer for the first 5 epochs to stabilize $\mathcal{L}_{\mathrm{ent}}$ decay; small trade-offs observed.
	\item \textbf{Span type probing}: Used auxiliary decoders (e.g., NER, chunking) to generate $\hat{P}_{\mathrm{gold}}$ in Equation~\eqref{eq:kl_span}; yielded slight gains in low-resource settings.
	\item \textbf{Pooling alternatives}: Replaced gated attention with mean or max pooling; gated attention retained superior semantic alignment (measured via cosine similarity to label embeddings).
\end{itemize}

\subsection{Vocabulary Induction Pseudocode}
\label{sec:induction-pseudocode}

\begin{algorithm}[H]
	\caption{Hybrid Unigram‐LM Vocabulary Induction (detailed)}
	\label{alg:induction-detailed}
	\begin{algorithmic}[1]
		\STATE Extract all substrings up to length $L_{\max}$; retain top $M$ by frequency and all codepoints
		\STATE Initialize $p^{(0)}(u) \propto \mathrm{freq}(u)$
		\STATE Compute baseline perplexity $\mathrm{PPL}^{(0)} = \exp(L^{(0)}/N_p^{(0)})$ via Viterbi over $\mathcal X$
		\FOR{$t=0$ \TO $T_{\max}$}
		\STATE \textbf{E-step:} Viterbi decode $\mathrm{seg}^*_t(x)$ for each $x$
		\STATE Collect counts $\gamma^{(t)}(u)\!=\!\sum_x\sum_{v\in\mathrm{seg}^*_t(x)}\mathbf1_{v=u}$
		\STATE \textbf{M-step:} $p^{(t+1)}(u)\!=\!\gamma^{(t)}(u)\big/\sum_v\gamma^{(t)}(v)$
		\FOR{each $u$ with $p^{(t+1)}(u)<\epsilon$}
		\STATE Form $V' = V\setminus\{u\}$ and Viterbi decode $\mathrm{seg}^*_{V'}(x)$
		\STATE Compute $L', N'_p, N'_{\mathrm{uncov}}$ as in Sec.~\ref{sec:vocab-induction}
		\IF{$\exp(L'/N'_p)-\mathrm{PPL}^{(t)}<\tau_{\mathrm{ppl}}$ and $N'_{\mathrm{uncov}}/N_t\le\delta_{\mathrm{oov}}$}
		\STATE Accept: $V\leftarrow V'$
		\ENDIF
		\ENDFOR
		\STATE Update $\mathrm{PPL}^{(t+1)}$ on current $V$
		\ENDFOR
		\STATE \textbf{Return} $V$ and $\{p(u)\}$
	\end{algorithmic}
\end{algorithm}

\subsection{Formal Proofs}
\label{sec:proofs}

\emph{Proof of Feasibility and Monotonicity} (Proposition in Sec.~\ref{sec:vocab-induction}).  
\[
V_0=\mathcal U_0\implies\mathrm{PPL}(V_0)=\mathrm{PPL}^{(0)},\;
\mathrm{OOV}(V_0)=0,
\]
so $V_0\in\mathcal F(\tau,\delta)$ for any $\tau,\delta\ge0$.  If $\tau'\ge\tau,\delta'\ge\delta$, then for $V\in\mathcal F(\tau,\delta)$,
\[
\mathrm{PPL}(V)\le \mathrm{PPL}^{(0)}+\tau\le \mathrm{PPL}^{(0)}+\tau',
\quad
\mathrm{OOV}(V)\le\delta\le\delta',
\quad\implies V\in\mathcal F(\tau',\delta'),
\]
establishing $\mathcal F(\tau,\delta)\subseteq\mathcal F(\tau',\delta')$ and the minimality relation.

\subsection{Dataset Construction and Statistics}
\label{sec:dataset}

We constructed three corpora:
\begin{itemize}[leftmargin=1.5em]
	\item \textbf{Code subset:} 10M lines of Python and JavaScript, avg.\ length 120 codepoints.
	\item \textbf{Multilingual subset:} 5M sentences across 10 languages, avg.\ 80 codepoints.
	\item \textbf{Hybrid subset:} 2M docs mixing code and prose, avg.\ 200 codepoints.
\end{itemize}
All datasets were cleaned for control characters and split 80/10/10 for train/val/test.

\subsection{ONNX Operator Implementation}
\label{sec:onnx-op}

The custom Unigram-LM operator is implemented in C++/CUDA for ONNX Runtime.  
\begin{itemize}[leftmargin=1.5em]
	\item \textbf{Inputs:} Raw codepoint tensor $[T]$, vocabulary table $\mathcal U_0$.
	\item \textbf{Outputs:} Sparse probability matrix $P\in\mathbb{R}^{T\times V}$.
	\item \textbf{Performance:} Processes 10\,000 tokens in 5ms on A100, including Viterbi decoding.
\end{itemize}

\subsection{Qualitative Segmentation Examples}
\label{sec:qual-examples}

\begin{table}[H]
	\centering
	\begin{tabular}{@{}p{0.4\linewidth}p{0.5\linewidth}@{}}
		\toprule
		Input & Top-3 Predicted Spans \\
		\midrule
		"def compute\_sum(x, y):"       & [def] [compute\_sum] [(x,] [y):] \\
		"<html><body>Welcome!"         & [<html>] [<body>] [Welcome] [!] \\
		"Bonjour, comment ça va ?"      & [Bonjour] [,] [comment] [ça] [va] [?] \\
		\bottomrule
	\end{tabular}
	\caption{Example segments with overlapping, variable-length spans and modality typing (code vs.\ punctuation vs.\ word).}
\end{table}

\subsection{Extended Evaluation Metrics}
\label{sec:extended-metrics}

In addition to the main metrics, we measured:
\begin{itemize}[leftmargin=1.5em]
	\item Memory footprint: peak GPU RAM during inference.
	\item Throughput: tokens/sec on batch size 32 across corpora.
	\item Segmentation consistency: IOU overlap of spans across runs with different seeds.
	\item Compression entropy: average per-token entropy before and after pruning.
\end{itemize}

\subsection{Judge Agent Prompt and Configuration}
\label{sec:judge-agent-config}

To automate quality filtering of OCR‐extracted segments, we deploy an LLM‐based judge agent. Below are the exact prompts and YAML configuration.

\paragraph{System Prompt}
\begin{verbatim}
	You are a data quality evaluator for X-Spanformer training data. You assess text segments
	for their suitability as training examples for a tokenizer-free, span-aware encoder.
	
	Evaluate segments based on:
	- Structural clarity: Can meaningful spans be identified (words, phrases, code constructs)?
	- Compositional value: Does it contain learnable patterns for span segmentation?
	- Training utility: Is it clean, coherent, and representative of target domains (code,
	natural language, or mixed)?
	
	Decision criteria:
	- Score >= {{ threshold }}: Status should be "keep"
	- Score <  {{ threshold }}: Status should be "discard"
	
	Output format (exactly 4 lines):
	Score: (float 0.0-1.0, where 1.0 = excellent training data)
	Status: keep | discard
	Type:   natural | code | mixed
	Reason: brief explanation focusing on structural/training value
	
	Be selective – only "keep" segments with clear structural patterns that will help the
	model learn span segmentation and composition.
\end{verbatim}

\paragraph{User Prompt}
\begin{verbatim}
	Evaluate this text segment as potential training data for X-Spanformer (tokenizer-free,
	span-aware model).
	
	Consider:
	- Does it have clear structural elements that can be segmented into meaningful spans?
	- Is it clean and well-formed for training purposes?
	- Does it represent valuable patterns for learning span composition?
	- What type of content is this?
	
	Decision criteria:
	- Score >= {{ threshold }}: Status should be "keep"
	- Score <  {{ threshold }}: Status should be "discard"
	
	Content types:
	- "natural": Natural language text (prose, articles, documentation)
	- "code":     Programming code, markup, configuration files
	- "mixed":    Combined natural language and code elements
	
	Text segment:
	---
	{{ text }}
	---
	
	Respond in exactly 4 lines:
	Score:  0.0-1.0
	Status: keep | discard
	Type:   natural | code | mixed
	Reason: brief structural assessment
\end{verbatim}

\paragraph{Judge Configuration (YAML)}
\begin{verbatim}
	agent_type: judge
	
	model:
	name: phi4-mini
	temperature: 0.2
	max_context_tokens: 131072
	max_turn_tokens: 4096
	
	processor:
	max_raw_length: 512
	
	dialogue:
	max_turns: 12
	memory_limit: 24
	trim_strategy: rolling
	
	judge:
	model_name: phi4-mini
	temperature: 0.2
	max_retries: 3
	judges: 5
	threshold: 0.69
	
	regex_filters:
	- pattern: "^[\\s\\n\\r]*$"
	reason: "empty or whitespace-only content"
	- pattern: "^[^a-zA-Z0-9]{10,}$"
	reason: "content with only symbols/punctuation"
	
	templates:
	system: judge_system
	judge:  segment_judge
	
	format:
	expected_fields: [ score, status, reason ]
	strict_line_count: 3
	parse_strategy:     regex
	
	logging:
	verbosity:            debug
	track_consensus:      true
	return_all_passes:    false
	log_queries:          true
	log_responses:        true
\end{verbatim}