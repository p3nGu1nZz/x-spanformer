\section{Related Work}

\subsection{Static Subword Tokenization}
Traditional transformer pipelines segment text into fixed subwords, most commonly via Byte-Pair Encoding (BPE) \cite{sennrich2016bpe} or unigram‐LM tokenizers such as SentencePiece \cite{kudo2018sentencepiece}.  These methods produce compact, efficient lookup tables and underpin many large-scale models \cite{vaswani2017attention,devlin2019bert,raffel2020t5}, but their immutable vocabularies (i) cannot adapt during training, (ii) fragment rare or compositional phrases under domain shift \cite{galle2021respite}, and (iii) require expensive re-tokenization when new syntax or terminology appear.

\subsection{Character- and Byte-Level Encoders}
To remove the subword bottleneck, several works operate directly on characters or bytes.  Charformer learns latent splits via gradient-based tokenization during training \cite{tay2021charformer}.  CANINE processes raw Unicode codepoints with down- and up-sampling layers, matching BPE baselines on transfer tasks \cite{clark2021canine}.  ByT5 further explores byte-level modeling for fully token-free representations \cite{xue2022byt5}.  These approaches eliminate offline heuristics but do not explicitly model overlapping or hierarchical spans.

\subsection{Differentiable Span Discovery}
Unsupervised segmentation methods integrate boundary induction into neural pretraining.  Morfessor uses minimum description‐length for morpheme discovery \cite{creutz2005unsupervised}.  Probabilistically Masked Language Models (PMLM) learn to mask and predict spans \cite{liu2022pmlm}, while other works optimize reconstruction losses to induce segmentation for text-to-text tasks \cite{liu2022learnedsegmentation}.  These differentiable methods yield end-to-end learning but generate non-overlapping partitions without explicit linguistic structure.

\subsection{Span-Based and Pointer-Network Models}
Pointer networks predict variable-length spans by emitting start/end indices \cite{vinyals2015pointer}.  SpanBERT applies this in pretraining by masking contiguous spans and reconstructing them \cite{joshi2020spanbert}.  In vision and speech, learned segmenters output overlapping proposals for improved alignment \cite{ren2015faster,zach2019segmenter}.  However, prior work does not integrate pointer-based, overlapping span prediction with generative grammar priors in transformer encoders.

\subsection{Bridging Static and Dynamic Segmentation}
Recent efforts seek a middle ground between rigid subwords and unstructured bytes.  Gradient-based BPE merges (Charformer) and hybrid vocabularies (static subwords plus characters) improve robustness \cite{tay2021charformer,clark2021canine}.  Our approach extends these ideas by training a hybrid Unigram-LM front-end—combining full‐coverage characters with entropy-pruned subwords—then learning overlapping, softly typed spans via a pointer network with length and modality priors.

\subsection{Summary}
Segmentation strategies span:

\begin{itemize}
	\item Offline subwords (BPE, Unigram) that are efficient but inflexible under domain shift \cite{sennrich2016bpe,kudo2018sentencepiece,galle2021respite}.
	\item Character- or byte-level encoders that remove heuristics but lack explicit higher-order units \cite{tay2021charformer,clark2021canine,xue2022byt5}.
	\item Differentiable segmentation that produces non-overlapping partitions without linguistic priors \cite{creutz2005unsupervised,liu2022learnedsegmentation,liu2022pmlm}.
	\item Span-based predictors that yield flexible proposals but have not been grounded in phrase structure \cite{vinyals2015pointer,joshi2020spanbert}.
\end{itemize}

X-Spanformer unifies pointer-based, overlapping span prediction with X-bar–inspired inductive biases, yielding dynamic, interpretable spans that integrate seamlessly into transformer encoders for end-to-end training.