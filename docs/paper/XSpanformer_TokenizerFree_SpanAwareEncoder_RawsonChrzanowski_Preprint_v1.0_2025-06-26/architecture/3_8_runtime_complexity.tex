\subsection{Runtime Complexity}

We analyze a single forward pass of X-Spanformer by decomposing it into six principal stages.  Let:
\[
T \;=\;\text{input length (codepoints)},\quad
V \;=\;\text{vocabulary size},\quad
d \;=\;\text{hidden dimension},\quad
w_{\max} \;=\;\text{maximum span width},\quad
K \;=\;\text{number of retained spans},\quad
d_f \;=\;\text{scoring‐MLP dimension}.
\]
Assume \(K\ll T\), \(w_{\max}\ll T\), and \(d_f=O(d)\).

\paragraph{1. Soft Segmentation \& Seed Embedding}
A custom Unigram‐LM operator (ONNX custom op) computes a sparse probability matrix
\(\,P\in\mathbb{R}^{T\times V}\) and multiplies by the embedding table \(W_{\mathrm{emb}}\in\mathbb{R}^{V\times d}\):
\[
P \;:\; O(TV), 
\quad
H^0 = P\,W_{\mathrm{emb}} \;:\; O(TV\,d).
\]
Since \(V\approx10^3\), this step costs \(O(T\,d)\) in practice.

\paragraph{2. Contextual Encoder}
Optionally, \(H^0\) is contextualized via a lightweight encoder (e.g., transformer or convnet):
\[
H = \mathrm{Encoder}(H^0) :\;O\bigl(T^2d\bigr)\quad(\text{transformer})\quad\text{or}\quad O\bigl(T\,d^2\bigr)\;(\text{conv}).
\]

\paragraph{3. Span Enumeration \& Boundary Scoring}
For each position \(i\), up to \(w_{\max}\) spans \((i,j)\) are scored by two linear heads on \(H\):
\[
\ell^s = W_s H + b_s,\quad \ell^e = W_e H + b_e :\;O(T\,d),
\quad
\text{score}(i,j)=p^s_i\,p^e_j : O(T\,w_{\max}).
\]
Total: \(O(T\,d + T\,w_{\max})\).

\paragraph{4. Span Embedding \& Scoring}
Selected top-\(K\) spans are pooled (\(\mathrm{Pool}\)) and optionally refined (self-attention):
\[
\mathrm{Pool}:O(K\,w_{\max}\,d),\quad
\mathrm{SelfAttn}:O(K\,w_{\max}^2\,d).
\]
Relevance logits via an MLP of size \(d_f\):
\[
w_k = f_{\mathrm{score}}(\cdot) : O(K\,d_f),\quad
\text{softmax}:O(K).
\]
Combined cost: \(O(K\,d + K\,d_f)\approx O(K\,d)\).

\paragraph{5. Controller Injection}
Computing the fused controller \(s=\sum_k a_k s_k\) costs \(O(K\,d)\).  Injecting \(s\) into the transformer adds at most
\[
O(T\,d)\quad(\text{bias or gated‐FFN})
\quad\text{or}\quad
O((T+1)d)\quad(\text{prefix token}),
\]
which is subsumed by the encoder cost.

\paragraph{6. Joint Contextualization}
Processing \(T\) tokens (plus one prefix) via dense attention costs
\[
O\bigl((T+\eta)^2\,d\bigr),
\quad \eta\in\{0,1\}.
\]

\begin{proposition}[Total Forward‐Pass Complexity]
	Under the above notation, X-Spanformer’s per‐example time is
	\[
	\Theta\bigl(TV\,d\;+\;T^2d\;+\;T\,w_{\max}\;+\;K\,d\;+\;(T+\eta)^2d\bigr).
	\]
	Absorbing constants and noting \(V, d\) fixed:
	\[
	=\;O\bigl(T\,w_{\max} + K\,d + T^2\,d\bigr).
	\]
\end{proposition}

\begin{proof}
	Summing stage costs:
	\[
	O(TVd) + O(T^2d) + O(Tw_{\max}) + O(Kd) + O(Kd) + O(Td) = O(T\,w_{\max}+K\,d+T^2d).
	\]
\end{proof}

This decomposition parallels sparse and routing Transformers such as Longformer \cite{beltagy2020longformer}, BigBird \cite{zaheer2020bigbird}, and MoE layers \cite{shazeer2017outrageously,ainslie2023transformers}, ensuring X-Spanformer scales subquadratically with sequence length for the segmentation stages and retains the quadratic self‐attention bound only in the final contextualization.