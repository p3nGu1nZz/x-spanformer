\subsection{Span Predictor}

Given the contextualized embeddings 
\[
H \;\in\;\mathbb{R}^{T\times d},
\]
we use two parallel linear heads (a factorized pointer network \cite{vinyals2015pointer}) to predict span boundaries. Concretely, we compute

\[
\ell^s \;=\; W_s H + b_s,\quad
p^s \;=\;\softmax(\ell^s), 
\qquad
\ell^e \;=\; W_e H + b_e,\quad
p^e \;=\;\softmax(\ell^e),
\]
where \(\ell^s,\ell^e\in\mathbb{R}^T\), and \(p^s_i\) (resp.\ \(p^e_j\)) is the probability that a span begins at position \(i\) (ends at \(j\)).  

Each candidate span \((i,j)\in C\) is scored by the product of its boundary probabilities:
\[
\mathrm{score}(i,j) 
\;=\; p^s_i \;\cdot\; p^e_j.
\]
This outer‐product scoring efficiently captures start–end salience and has been widely used in QA and entity extraction \cite{lee2016learning,xu2022faster}.  

We then select the top-\(K\) spans by marginal likelihood:
\[
S 
= \TopK\bigl\{\mathrm{score}(i,j)\mid (i,j)\in C\bigr\}.
\]

\begin{proposition}[Top-\(K\) Marginal Likelihood]
	Let \(p^s,p^e\in\Delta^T\) be independent start/end distributions over \(T\) positions.  Define 
	\[
	P(i,j)\;=\;p^s_i\,p^e_j
	\quad
	\text{for all }(i,j)\in C=\{1\le i<j\le T\}.
	\]
	Then the set 
	\(
	S=\TopK\{P(i,j)\}
	\)
	maximizes the total probability mass over all \(K\)-sized span subsets:
	\[
	S
	\;=\;
	\arg\max_{\substack{S'\subseteq C\\|S'|=K}}
	\sum_{(i,j)\in S'}P(i,j).
	\]
\end{proposition}

\begin{proof}
	Since \(P(i,j)\ge0\) and additive, greedily selecting the top \(K\) values of \(P(i,j)\) maximizes \(\sum_{(i,j)\in S'}P(i,j)\).  Independence ensures no additional interaction terms.
\end{proof}